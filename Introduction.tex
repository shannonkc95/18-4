Google's PageRank algorithm, developed in 1998 by Sergey Brin and Larry Page is one of the most well-known and influential methods of ranking web pages. The hyperlink structure of the web is represented by a directed graph, with the web pages representing the nodes and the links between nodes being the hyperlinks. Inlinks are pages that point into nodes, and outlinks point out from nodes \cite{brin1998anatomy}. The algorithm in its simplest form aims to rank pages with the assumption that a web page's importance is reliant on other important pages linking to it. When this is represented mathematically, it becomes clear that the importance scores of the web pages relate to the stationary points of a Markov chain, and hence Markov theory and matrices can be used to solve the PageRank equation:
\begin{equation}
\boldsymbol{\pi} = \boldsymbol{\pi} \cdot \textbf{G}
\end{equation}
where \textbf{G} is the 'Google Matrix' and $\boldsymbol{\pi}$ is the set of ranked web pages \cite{langville}. We are able to interpret a web pages PageRank as the fraction of time that a random surfer spends on that web page, with the random surfer more likely to return to the most important pages \cite{bonato}. Throughout this report we will be using the network graph as an example: 
\begin{figure}[H]
\centering
\begin{tikzpicture}[->,>=stealth',shorten >=1pt,auto,node distance=2cm,
                    semithick]
\tikzstyle{every state}=[fill=white,draw=black,text=black]
\node[state] (1) {$1$};
\node[state] (2) [right of=1] {$2$};
\node[state] (3) [right of=2] {$3$};
\node[state] (4) [below of=1] {$4$};
\node[state] (5) [below of=3] {$5$};
\node[state] (6) [below of=4] {$6$};
\node[state] (7) [right of=6] {$7$};
\node[state] (8) [right of=7] {$8$};
\path (1) edge node{} (2)
          edge node{} (4)
          edge [bend right] node{} (6)
      (2) edge node{} (3)
          edge node{} (5)
      (3) edge node{} (5)
      (4) edge node{} (7)
          edge node{} (5)
      (5) edge [bend right] node{} (8)
      (6) edge node{} (4)
          edge node{} (7)
          edge [bend right] node{} (8)
      (8) edge [bend right] node{} (5)
          edge node{} (7);
          
\end{tikzpicture}
\caption{Graph modelling 8-node web} \label{fig:Example}
\end{figure}

In the first chapter of this report, we will explore mathematical representations of the PageRank equation, the modifications required on the matrix in order to use Markov theory to solve the PageRank equation, and finally on the use of the power method to solve the equation. Chapter \ref{chap:Other} discusses other methods of ranking web pages, and compares these to PageRank. Chapter \ref{chap:Improve} will then explore improvements to the PageRank representation that could increase the accuracy and personalise the PageRank method for a user, such as weighted PageRank and topic-specific PageRank. Chapter \ref{chap:Applications} explores potential applications for PageRank beyond ranking web-pages, for example in predicting traffic flow in Durham. The final chapter of the report, Chapter \ref{chap:Conclusion}, summarises the main points of the report.
