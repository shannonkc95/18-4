\chapter{Matlab code} \label{app:code}

\begin{lstlisting} [caption=Matlab code for implementing power method taken from \cite{langville} ]
function [pi,time,numiter] = \hbox{PageRank}(pi0,H,n,alpha,epsilon);

% \hbox{PageRank} computes the \hbox{PageRank} vector for an n-by-n Markov
%           matrix H with starting vector pi0 (a row vector)
%           and scaling parameter alpha (scalar). Uses power method.
%
% EXAMPLE: [pi,time,numiter] = \hbox{PageRank}(pi0,H,1000,.9,1e-8)
%
% INPUT: pi0 = starting vector at iteration 0 (row vector)
%        H = row-normalized hyperlink matrix (n-by-n sparse matrix)
%        n = size of H matrix (scalar)
%        alpha = scaling parameter in \hbox{PageRank} model (scalar)
%        epsilon = convergence tolerance (scalar, e.g. 1e-8)
%
% OUTPUT: pi = \hbox{PageRank} vector
%         time = time required to compute \hbox{PageRank} vector
%         numiter = number of iterations until convergence
%        
% Starting vector is usually set to uniform vector,
% pi0=1/n*ones(1,n).
% NOTE: Matlab stores sparse matrices by columns, so it is faster
%       to do some operations on H', the transpose of H.
% get 'a', dangling node vector, where a(i) = 1, if node i 
%    is dangling node and 0 o.w.

rowsumvector = ones(1,n)*H';
nonzerorows=find(rowsumvector);
zerorows=setdiff(1:n,nonzerorows); l=length(zerorows);
a=sparse(zerorows,ones(1,1),ones(1,1),n,1);

k=0;
residual=1;
pi=pi0;
tic;

while(residual>=epsilon)
     prevpi=pi;
     k=k+1;
     pi=alpha*pi*H + (alpha*(pi*a)+1-alpha)*((1/n)*ones(1,n));
     residual=norm(pi-prevpi,1);
end

numiter=k;
time=toc;

\end{lstlisting}

\begin{lstlisting} [caption=Matlab code for implementing HITS power method taken from \cite{langville}]
function [x,y,time,numiter] = hits[L,x0,n,epsilon] 

% HITS computes the HITS authority vector x and hub vector y 
%      for an n-by-n adjacency matrix L with starting vector
%      x0 (a row vector). Uses power method on L'*L.
%
% EXAMPLE: [x,y,time,numiter] = hits[L,x0,100,1e-8]
%
% INPUT: L = adjacency matrix (n-by-n sparse matrix)
%        x0 = starting vector (row vector)
%        n = size of L matrix (integer)
%        epsilon = convergence tolerance (scalar, e.g. 1e-8)
%
% OUTPUT: x = HITS authority vector
%         y = HITS hub vector
%         time = time until convergence
%         numiter = number of iterations until convergence
%        
% The starting vector is usually set to the uniform vector,
% x0=1/n*ones(1/n)

k=0;
residual=1;
x=x0;
tic;

while(residual>=epsilon)
     prevx=x;
     k=k+1;
     x=x*L';
     x=x*L;
     x=x/sum(x);
     residual=norm(x-prevx,1);
end
y=x*L';
y=y/sum(y);
numiter=k;
time=toc;

\end{lstlisting}