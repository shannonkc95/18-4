\section{Wikipedia} \label{sec:wiki}
We are able to apply PR to large databases such as Wikipedia, for example, we can generate reading lists for university courses automatically from exploiting the link structure of Wikipedia \cite{wissner2006preparation}. We are able to rank articles to find the top 100 historical figures and which cultures have the largest reach, for example a culture will have a high PR if it has many inlinks from other cultures. PR reveals that Carl Linnaeus is the most 'important' page due to the fact that he developed the scientific classification system we use for animals, insects and plants, and so these pages link back to him \cite{eom2015interactions}. We are also able to conclude that the most important figures in history are Western men who are born after the 17th Century. This mathematical analysis of historical figures and cultures using Wikipedia can be very useful in exploring interactions between world cultures and understanding history.

